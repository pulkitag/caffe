


% Header, overrides base

    % Make sure that the sphinx doc style knows who it inherits from.
    \def\sphinxdocclass{article}

    % Declare the document class
    \documentclass[letterpaper,10pt,english]{/usr/share/sphinx/texinputs/sphinxhowto}

    % Imports
    \usepackage[utf8]{inputenc}
    \DeclareUnicodeCharacter{00A0}{\\nobreakspace}
    \usepackage[T1]{fontenc}
    \usepackage{babel}
    \usepackage{times}
    \usepackage{import}
    \usepackage[Bjarne]{/usr/share/sphinx/texinputs/fncychap}
    \usepackage{longtable}
    \usepackage{/usr/share/sphinx/texinputs/sphinx}
    \usepackage{multirow}

    \usepackage{amsmath}
    \usepackage{amssymb}
    \usepackage{ucs}
    \usepackage{enumerate}

    % Used to make the Input/Output rules follow around the contents.
    \usepackage{needspace}

    % Pygments requirements
    \usepackage{fancyvrb}
    \usepackage{color}
    % ansi colors additions
    \definecolor{darkgreen}{rgb}{.12,.54,.11}
    \definecolor{lightgray}{gray}{.95}
    \definecolor{brown}{rgb}{0.54,0.27,0.07}
    \definecolor{purple}{rgb}{0.5,0.0,0.5}
    \definecolor{darkgray}{gray}{0.25}
    \definecolor{lightred}{rgb}{1.0,0.39,0.28}
    \definecolor{lightgreen}{rgb}{0.48,0.99,0.0}
    \definecolor{lightblue}{rgb}{0.53,0.81,0.92}
    \definecolor{lightpurple}{rgb}{0.87,0.63,0.87}
    \definecolor{lightcyan}{rgb}{0.5,1.0,0.83}

    % Needed to box output/input
    \usepackage{tikz}
        \usetikzlibrary{calc,arrows,shadows}
    \usepackage[framemethod=tikz]{mdframed}

    \usepackage{alltt}

    % Used to load and display graphics
    \usepackage{graphicx}
    \graphicspath{ {figs/} }
    \usepackage[Export]{adjustbox} % To resize

    % used so that images for notebooks which have spaces in the name can still be included
    \usepackage{grffile}


    % For formatting output while also word wrapping.
    \usepackage{listings}
    \lstset{breaklines=true}
    \lstset{basicstyle=\small\ttfamily}
    \def\smaller{\fontsize{9.5pt}{9.5pt}\selectfont}

    %Pygments definitions
    
\makeatletter
\def\PY@reset{\let\PY@it=\relax \let\PY@bf=\relax%
    \let\PY@ul=\relax \let\PY@tc=\relax%
    \let\PY@bc=\relax \let\PY@ff=\relax}
\def\PY@tok#1{\csname PY@tok@#1\endcsname}
\def\PY@toks#1+{\ifx\relax#1\empty\else%
    \PY@tok{#1}\expandafter\PY@toks\fi}
\def\PY@do#1{\PY@bc{\PY@tc{\PY@ul{%
    \PY@it{\PY@bf{\PY@ff{#1}}}}}}}
\def\PY#1#2{\PY@reset\PY@toks#1+\relax+\PY@do{#2}}

\expandafter\def\csname PY@tok@gd\endcsname{\def\PY@tc##1{\textcolor[rgb]{0.63,0.00,0.00}{##1}}}
\expandafter\def\csname PY@tok@gu\endcsname{\let\PY@bf=\textbf\def\PY@tc##1{\textcolor[rgb]{0.50,0.00,0.50}{##1}}}
\expandafter\def\csname PY@tok@gt\endcsname{\def\PY@tc##1{\textcolor[rgb]{0.00,0.27,0.87}{##1}}}
\expandafter\def\csname PY@tok@gs\endcsname{\let\PY@bf=\textbf}
\expandafter\def\csname PY@tok@gr\endcsname{\def\PY@tc##1{\textcolor[rgb]{1.00,0.00,0.00}{##1}}}
\expandafter\def\csname PY@tok@cm\endcsname{\let\PY@it=\textit\def\PY@tc##1{\textcolor[rgb]{0.25,0.50,0.50}{##1}}}
\expandafter\def\csname PY@tok@vg\endcsname{\def\PY@tc##1{\textcolor[rgb]{0.10,0.09,0.49}{##1}}}
\expandafter\def\csname PY@tok@m\endcsname{\def\PY@tc##1{\textcolor[rgb]{0.40,0.40,0.40}{##1}}}
\expandafter\def\csname PY@tok@mh\endcsname{\def\PY@tc##1{\textcolor[rgb]{0.40,0.40,0.40}{##1}}}
\expandafter\def\csname PY@tok@go\endcsname{\def\PY@tc##1{\textcolor[rgb]{0.53,0.53,0.53}{##1}}}
\expandafter\def\csname PY@tok@ge\endcsname{\let\PY@it=\textit}
\expandafter\def\csname PY@tok@vc\endcsname{\def\PY@tc##1{\textcolor[rgb]{0.10,0.09,0.49}{##1}}}
\expandafter\def\csname PY@tok@il\endcsname{\def\PY@tc##1{\textcolor[rgb]{0.40,0.40,0.40}{##1}}}
\expandafter\def\csname PY@tok@cs\endcsname{\let\PY@it=\textit\def\PY@tc##1{\textcolor[rgb]{0.25,0.50,0.50}{##1}}}
\expandafter\def\csname PY@tok@cp\endcsname{\def\PY@tc##1{\textcolor[rgb]{0.74,0.48,0.00}{##1}}}
\expandafter\def\csname PY@tok@gi\endcsname{\def\PY@tc##1{\textcolor[rgb]{0.00,0.63,0.00}{##1}}}
\expandafter\def\csname PY@tok@gh\endcsname{\let\PY@bf=\textbf\def\PY@tc##1{\textcolor[rgb]{0.00,0.00,0.50}{##1}}}
\expandafter\def\csname PY@tok@ni\endcsname{\let\PY@bf=\textbf\def\PY@tc##1{\textcolor[rgb]{0.60,0.60,0.60}{##1}}}
\expandafter\def\csname PY@tok@nl\endcsname{\def\PY@tc##1{\textcolor[rgb]{0.63,0.63,0.00}{##1}}}
\expandafter\def\csname PY@tok@nn\endcsname{\let\PY@bf=\textbf\def\PY@tc##1{\textcolor[rgb]{0.00,0.00,1.00}{##1}}}
\expandafter\def\csname PY@tok@no\endcsname{\def\PY@tc##1{\textcolor[rgb]{0.53,0.00,0.00}{##1}}}
\expandafter\def\csname PY@tok@na\endcsname{\def\PY@tc##1{\textcolor[rgb]{0.49,0.56,0.16}{##1}}}
\expandafter\def\csname PY@tok@nb\endcsname{\def\PY@tc##1{\textcolor[rgb]{0.00,0.50,0.00}{##1}}}
\expandafter\def\csname PY@tok@nc\endcsname{\let\PY@bf=\textbf\def\PY@tc##1{\textcolor[rgb]{0.00,0.00,1.00}{##1}}}
\expandafter\def\csname PY@tok@nd\endcsname{\def\PY@tc##1{\textcolor[rgb]{0.67,0.13,1.00}{##1}}}
\expandafter\def\csname PY@tok@ne\endcsname{\let\PY@bf=\textbf\def\PY@tc##1{\textcolor[rgb]{0.82,0.25,0.23}{##1}}}
\expandafter\def\csname PY@tok@nf\endcsname{\def\PY@tc##1{\textcolor[rgb]{0.00,0.00,1.00}{##1}}}
\expandafter\def\csname PY@tok@si\endcsname{\let\PY@bf=\textbf\def\PY@tc##1{\textcolor[rgb]{0.73,0.40,0.53}{##1}}}
\expandafter\def\csname PY@tok@s2\endcsname{\def\PY@tc##1{\textcolor[rgb]{0.73,0.13,0.13}{##1}}}
\expandafter\def\csname PY@tok@vi\endcsname{\def\PY@tc##1{\textcolor[rgb]{0.10,0.09,0.49}{##1}}}
\expandafter\def\csname PY@tok@nt\endcsname{\let\PY@bf=\textbf\def\PY@tc##1{\textcolor[rgb]{0.00,0.50,0.00}{##1}}}
\expandafter\def\csname PY@tok@nv\endcsname{\def\PY@tc##1{\textcolor[rgb]{0.10,0.09,0.49}{##1}}}
\expandafter\def\csname PY@tok@s1\endcsname{\def\PY@tc##1{\textcolor[rgb]{0.73,0.13,0.13}{##1}}}
\expandafter\def\csname PY@tok@sh\endcsname{\def\PY@tc##1{\textcolor[rgb]{0.73,0.13,0.13}{##1}}}
\expandafter\def\csname PY@tok@sc\endcsname{\def\PY@tc##1{\textcolor[rgb]{0.73,0.13,0.13}{##1}}}
\expandafter\def\csname PY@tok@sx\endcsname{\def\PY@tc##1{\textcolor[rgb]{0.00,0.50,0.00}{##1}}}
\expandafter\def\csname PY@tok@bp\endcsname{\def\PY@tc##1{\textcolor[rgb]{0.00,0.50,0.00}{##1}}}
\expandafter\def\csname PY@tok@c1\endcsname{\let\PY@it=\textit\def\PY@tc##1{\textcolor[rgb]{0.25,0.50,0.50}{##1}}}
\expandafter\def\csname PY@tok@kc\endcsname{\let\PY@bf=\textbf\def\PY@tc##1{\textcolor[rgb]{0.00,0.50,0.00}{##1}}}
\expandafter\def\csname PY@tok@c\endcsname{\let\PY@it=\textit\def\PY@tc##1{\textcolor[rgb]{0.25,0.50,0.50}{##1}}}
\expandafter\def\csname PY@tok@mf\endcsname{\def\PY@tc##1{\textcolor[rgb]{0.40,0.40,0.40}{##1}}}
\expandafter\def\csname PY@tok@err\endcsname{\def\PY@bc##1{\setlength{\fboxsep}{0pt}\fcolorbox[rgb]{1.00,0.00,0.00}{1,1,1}{\strut ##1}}}
\expandafter\def\csname PY@tok@kd\endcsname{\let\PY@bf=\textbf\def\PY@tc##1{\textcolor[rgb]{0.00,0.50,0.00}{##1}}}
\expandafter\def\csname PY@tok@ss\endcsname{\def\PY@tc##1{\textcolor[rgb]{0.10,0.09,0.49}{##1}}}
\expandafter\def\csname PY@tok@sr\endcsname{\def\PY@tc##1{\textcolor[rgb]{0.73,0.40,0.53}{##1}}}
\expandafter\def\csname PY@tok@mo\endcsname{\def\PY@tc##1{\textcolor[rgb]{0.40,0.40,0.40}{##1}}}
\expandafter\def\csname PY@tok@kn\endcsname{\let\PY@bf=\textbf\def\PY@tc##1{\textcolor[rgb]{0.00,0.50,0.00}{##1}}}
\expandafter\def\csname PY@tok@mi\endcsname{\def\PY@tc##1{\textcolor[rgb]{0.40,0.40,0.40}{##1}}}
\expandafter\def\csname PY@tok@gp\endcsname{\let\PY@bf=\textbf\def\PY@tc##1{\textcolor[rgb]{0.00,0.00,0.50}{##1}}}
\expandafter\def\csname PY@tok@o\endcsname{\def\PY@tc##1{\textcolor[rgb]{0.40,0.40,0.40}{##1}}}
\expandafter\def\csname PY@tok@kr\endcsname{\let\PY@bf=\textbf\def\PY@tc##1{\textcolor[rgb]{0.00,0.50,0.00}{##1}}}
\expandafter\def\csname PY@tok@s\endcsname{\def\PY@tc##1{\textcolor[rgb]{0.73,0.13,0.13}{##1}}}
\expandafter\def\csname PY@tok@kp\endcsname{\def\PY@tc##1{\textcolor[rgb]{0.00,0.50,0.00}{##1}}}
\expandafter\def\csname PY@tok@w\endcsname{\def\PY@tc##1{\textcolor[rgb]{0.73,0.73,0.73}{##1}}}
\expandafter\def\csname PY@tok@kt\endcsname{\def\PY@tc##1{\textcolor[rgb]{0.69,0.00,0.25}{##1}}}
\expandafter\def\csname PY@tok@ow\endcsname{\let\PY@bf=\textbf\def\PY@tc##1{\textcolor[rgb]{0.67,0.13,1.00}{##1}}}
\expandafter\def\csname PY@tok@sb\endcsname{\def\PY@tc##1{\textcolor[rgb]{0.73,0.13,0.13}{##1}}}
\expandafter\def\csname PY@tok@k\endcsname{\let\PY@bf=\textbf\def\PY@tc##1{\textcolor[rgb]{0.00,0.50,0.00}{##1}}}
\expandafter\def\csname PY@tok@se\endcsname{\let\PY@bf=\textbf\def\PY@tc##1{\textcolor[rgb]{0.73,0.40,0.13}{##1}}}
\expandafter\def\csname PY@tok@sd\endcsname{\let\PY@it=\textit\def\PY@tc##1{\textcolor[rgb]{0.73,0.13,0.13}{##1}}}

\def\PYZbs{\char`\\}
\def\PYZus{\char`\_}
\def\PYZob{\char`\{}
\def\PYZcb{\char`\}}
\def\PYZca{\char`\^}
\def\PYZam{\char`\&}
\def\PYZlt{\char`\<}
\def\PYZgt{\char`\>}
\def\PYZsh{\char`\#}
\def\PYZpc{\char`\%}
\def\PYZdl{\char`\$}
\def\PYZhy{\char`\-}
\def\PYZsq{\char`\'}
\def\PYZdq{\char`\"}
\def\PYZti{\char`\~}
% for compatibility with earlier versions
\def\PYZat{@}
\def\PYZlb{[}
\def\PYZrb{]}
\makeatother


    %Set pygments styles if needed...
    
        \definecolor{nbframe-border}{rgb}{0.867,0.867,0.867}
        \definecolor{nbframe-bg}{rgb}{0.969,0.969,0.969}
        \definecolor{nbframe-in-prompt}{rgb}{0.0,0.0,0.502}
        \definecolor{nbframe-out-prompt}{rgb}{0.545,0.0,0.0}

        \newenvironment{ColorVerbatim}
        {\begin{mdframed}[%
            roundcorner=1.0pt, %
            backgroundcolor=nbframe-bg, %
            userdefinedwidth=1\linewidth, %
            leftmargin=0.1\linewidth, %
            innerleftmargin=0pt, %
            innerrightmargin=0pt, %
            linecolor=nbframe-border, %
            linewidth=1pt, %
            usetwoside=false, %
            everyline=true, %
            innerlinewidth=3pt, %
            innerlinecolor=nbframe-bg, %
            middlelinewidth=1pt, %
            middlelinecolor=nbframe-bg, %
            outerlinewidth=0.5pt, %
            outerlinecolor=nbframe-border, %
            needspace=0pt
        ]}
        {\end{mdframed}}
        
        \newenvironment{InvisibleVerbatim}
        {\begin{mdframed}[leftmargin=0.1\linewidth,innerleftmargin=3pt,innerrightmargin=3pt, userdefinedwidth=1\linewidth, linewidth=0pt, linecolor=white, usetwoside=false]}
        {\end{mdframed}}

        \renewenvironment{Verbatim}[1][\unskip]
        {\begin{alltt}\smaller}
        {\end{alltt}}
    

    % Help prevent overflowing lines due to urls and other hard-to-break 
    % entities.  This doesn't catch everything...
    \sloppy

    % Document level variables
    \title{Kitti-Document}
    \date{April 3, 2015}
    \release{}
    \author{Unknown Author}
    \renewcommand{\releasename}{}

    % TODO: Add option for the user to specify a logo for his/her export.
    \newcommand{\sphinxlogo}{}

    % Make the index page of the document.
    \makeindex

    % Import sphinx document type specifics.
     


% Body

    % Start of the document
    \begin{document}

        
            \maketitle
        

        


        
        Introduction

The goal is learning features using ego-motion/odometry.

Kitti provides data of a car moving aroud with a camera and records the
movements made by the car. For each frame the transformation with
respect to frame number 1 of the sequence is provided. There are 11
total sequences for which data is present. The transformation is
provided as a 3x4 transformation matrix. I represent transformation as
translation + three euler angles.

My goal is to train a network to predict these 6 transformation
parameters from pairs of images. I can either regress to these 6
parameters or I can bin these transformation and train a classification
network. Regressing the rotations has an issue that rotation of
360degrees is same as 0 degrees. But if the rotations are small then
this is not an issue. Currently, my plan is to regress.

If my network is not working - then one thing I can try is to use
optical flow and then try to regress on these variables.Labels of the Data

For understanding how the data looks like, I created a visualization of
the labels. The three plots contain data for relative translation and
rotations between consequent frames and absolute translation between
frames as compared to the first frame. Each vertical grey line indicates
the end of a sequence. There are 11 sequences overall. As expected, the
most translation changes are along (X,Z) directions (The camera points
in the Z direction) and rotations are about the Y axis.

    % Make sure that atleast 4 lines are below the HR
    \needspace{4\baselineskip}

    
        \vspace{6pt}
        \makebox[0.1\linewidth]{\smaller\hfill\tt\color{nbframe-in-prompt}In\hspace{4pt}{[}1{]}:\hspace{4pt}}\\*
        \vspace{-2.65\baselineskip}
        \begin{ColorVerbatim}
            \vspace{-0.7\baselineskip}
            \begin{Verbatim}[commandchars=\\\{\}]
\PY{o}{\PYZpc{}}\PY{k}{matplotlib} \PY{n}{inline}
\PY{o}{\PYZpc{}}\PY{k}{load\PYZus{}ext} \PY{n}{autoreload}
\PY{o}{\PYZpc{}}\PY{k}{autoreload} \PY{l+m+mi}{2}
\PY{k+kn}{import} \PY{n+nn}{os}
\PY{n}{kittiDir} \PY{o}{=} \PY{l+s}{\PYZsq{}}\PY{l+s}{/work4/pulkitag\PYZhy{}code/pkgs/caffe\PYZhy{}v2\PYZhy{}2/modelFiles/kitti/codes}\PY{l+s}{\PYZsq{}}
\PY{n}{os}\PY{o}{.}\PY{n}{chdir}\PY{p}{(}\PY{n}{kittiDir}\PY{p}{)}
\PY{k+kn}{import} \PY{n+nn}{kitti\PYZus{}utils} \PY{k+kn}{as} \PY{n+nn}{ku}
\PY{n}{ku}\PY{o}{.}\PY{n}{plot\PYZus{}pose}\PY{p}{(}\PY{n}{seqNum}\PY{o}{=}\PY{l+s}{\PYZsq{}}\PY{l+s}{all}\PY{l+s}{\PYZsq{}}\PY{p}{,} \PY{n}{poseType}\PY{o}{=}\PY{l+s}{\PYZsq{}}\PY{l+s}{euler}\PY{l+s}{\PYZsq{}}\PY{p}{)}
\end{Verbatim}

            
                \vspace{-0.2\baselineskip}
            
        \end{ColorVerbatim}
    

    

        % If the first block is an image, minipage the image.  Else
        % request a certain amount of space for the input text.
        \needspace{4\baselineskip}
        
        

            % Add document contents.
            
                \begin{InvisibleVerbatim}
                \vspace{-0.5\baselineskip}
    \begin{center}
    \includegraphics[max size={\textwidth}{\textheight}]{Kitti-Document_files/Kitti-Document_2_0.png}
    \par
    \end{center}
    
            \end{InvisibleVerbatim}
            
                \begin{InvisibleVerbatim}
                \vspace{-0.5\baselineskip}
    \begin{center}
    \includegraphics[max size={\textwidth}{\textheight}]{Kitti-Document_files/Kitti-Document_2_1.png}
    \par
    \end{center}
    
            \end{InvisibleVerbatim}
            
                \begin{InvisibleVerbatim}
                \vspace{-0.5\baselineskip}
    \begin{center}
    \includegraphics[max size={\textwidth}{\textheight}]{Kitti-Document_files/Kitti-Document_2_2.png}
    \par
    \end{center}
    
            \end{InvisibleVerbatim}
            
        
    
Initialization from Scratch

Weights of the network trained from scratch after 30,000 iterations. The
network overfits. The training error is very small, but test error is
quite large. (See the plots below).

    % Make sure that atleast 4 lines are below the HR
    \needspace{4\baselineskip}

    
        \vspace{6pt}
        \makebox[0.1\linewidth]{\smaller\hfill\tt\color{nbframe-in-prompt}In\hspace{4pt}{[}31{]}:\hspace{4pt}}\\*
        \vspace{-2.65\baselineskip}
        \begin{ColorVerbatim}
            \vspace{-0.7\baselineskip}
            \begin{Verbatim}[commandchars=\\\{\}]
\PY{k+kn}{import} \PY{n+nn}{my\PYZus{}pycaffe} \PY{k+kn}{as} \PY{n+nn}{mp}
\PY{k+kn}{import} \PY{n+nn}{matplotlib.pyplot} \PY{k+kn}{as} \PY{n+nn}{plt}
\PY{n}{mp} \PY{o}{=} \PY{n+nb}{reload}\PY{p}{(}\PY{n}{mp}\PY{p}{)}
\PY{n}{ax1} \PY{o}{=} \PY{n}{plt}\PY{o}{.}\PY{n}{subplot}\PY{p}{(}\PY{l+m+mi}{1}\PY{p}{,}\PY{l+m+mi}{2}\PY{p}{,}\PY{l+m+mi}{1}\PY{p}{)}
\PY{n}{ax2} \PY{o}{=} \PY{n}{plt}\PY{o}{.}\PY{n}{subplot}\PY{p}{(}\PY{l+m+mi}{1}\PY{p}{,}\PY{l+m+mi}{2}\PY{p}{,}\PY{l+m+mi}{2}\PY{p}{)}
\PY{n}{weightFile}\PY{p}{,} \PY{n}{protoFile} \PY{o}{=} \PY{n}{ku}\PY{o}{.}\PY{n}{get\PYZus{}weight\PYZus{}proto\PYZus{}file}\PY{p}{(}\PY{n}{isScratch}\PY{o}{=}\PY{n+nb+bp}{True}\PY{p}{,} \PY{n}{numIter}\PY{o}{=}\PY{l+m+mi}{30000}\PY{p}{)}
\PY{n}{net} \PY{o}{=} \PY{n}{mp}\PY{o}{.}\PY{n}{MyNet}\PY{p}{(}\PY{n}{protoFile}\PY{p}{,} \PY{n}{weightFile}\PY{p}{)}
\PY{n}{net}\PY{o}{.}\PY{n}{vis\PYZus{}weights}\PY{p}{(}\PY{l+s}{\PYZsq{}}\PY{l+s}{conv1}\PY{l+s}{\PYZsq{}}\PY{p}{,} \PY{n}{ax}\PY{o}{=}\PY{n}{ax1}\PY{p}{,} \PY{n}{titleName}\PY{o}{=}\PY{l+s}{\PYZsq{}}\PY{l+s}{Kitti Weights}\PY{l+s}{\PYZsq{}}\PY{p}{)}
\PY{n}{netRandom} \PY{o}{=} \PY{n}{mp}\PY{o}{.}\PY{n}{MyNet}\PY{p}{(}\PY{n}{protoFile}\PY{p}{)}
\PY{n}{netRandom}\PY{o}{.}\PY{n}{vis\PYZus{}weights}\PY{p}{(}\PY{l+s}{\PYZsq{}}\PY{l+s}{conv1}\PY{l+s}{\PYZsq{}}\PY{p}{,} \PY{n}{ax}\PY{o}{=}\PY{n}{ax2}\PY{p}{,} \PY{n}{titleName}\PY{o}{=}\PY{l+s}{\PYZsq{}}\PY{l+s}{Random Intialization}\PY{l+s}{\PYZsq{}}\PY{p}{)}
\end{Verbatim}

            
                \vspace{-0.2\baselineskip}
            
        \end{ColorVerbatim}
    

    

        % If the first block is an image, minipage the image.  Else
        % request a certain amount of space for the input text.
        \needspace{4\baselineskip}
        
        

            % Add document contents.
            
                \begin{InvisibleVerbatim}
                \vspace{-0.5\baselineskip}
    \begin{center}
    \includegraphics[max size={\textwidth}{\textheight}]{Kitti-Document_files/Kitti-Document_4_0.png}
    \par
    \end{center}
    
            \end{InvisibleVerbatim}
            
        
    


    % Make sure that atleast 4 lines are below the HR
    \needspace{4\baselineskip}

    
        \vspace{6pt}
        \makebox[0.1\linewidth]{\smaller\hfill\tt\color{nbframe-in-prompt}In\hspace{4pt}{[}22{]}:\hspace{4pt}}\\*
        \vspace{-2.65\baselineskip}
        \begin{ColorVerbatim}
            \vspace{-0.7\baselineskip}
            \begin{Verbatim}[commandchars=\\\{\}]
\PY{n}{ku}\PY{o}{.}\PY{n}{get\PYZus{}accuracy}\PY{p}{(}\PY{n}{numIter}\PY{o}{=}\PY{l+m+mi}{30000}\PY{p}{,} \PY{n}{imSz}\PY{o}{=}\PY{l+m+mi}{256}\PY{p}{,} \PY{n}{poseType}\PY{o}{=}\PY{l+s}{\PYZsq{}}\PY{l+s}{euler}\PY{l+s}{\PYZsq{}}\PY{p}{,} \PY{n}{nrmlzType}\PY{o}{=}\PY{l+s}{\PYZsq{}}\PY{l+s}{zScoreScaleSeperate}\PY{l+s}{\PYZsq{}}\PY{p}{,}
                 \PY{n}{isScratch}\PY{o}{=}\PY{n+nb+bp}{True}\PY{p}{,} \PY{n}{concatLayer}\PY{o}{=}\PY{l+s}{\PYZsq{}}\PY{l+s}{pool5}\PY{l+s}{\PYZsq{}}\PY{p}{,} \PY{n}{numBatches}\PY{o}{=}\PY{l+m+mi}{3}\PY{p}{)}
\end{Verbatim}

            
                \vspace{-0.2\baselineskip}
            
        \end{ColorVerbatim}
    

    

        % If the first block is an image, minipage the image.  Else
        % request a certain amount of space for the input text.
        \needspace{4\baselineskip}
        
        

            % Add document contents.
            
                \begin{InvisibleVerbatim}
                \vspace{-0.5\baselineskip}
\begin{alltt}Intializing Network
Calculating Features
Plotting Results
\end{alltt}

            \end{InvisibleVerbatim}
            
                \begin{InvisibleVerbatim}
                \vspace{-0.5\baselineskip}
    \begin{center}
    \includegraphics[max size={\textwidth}{\textheight}]{Kitti-Document_files/Kitti-Document_5_1.png}
    \par
    \end{center}
    
            \end{InvisibleVerbatim}
            
                \begin{InvisibleVerbatim}
                \vspace{-0.5\baselineskip}
    \begin{center}
    \includegraphics[max size={\textwidth}{\textheight}]{Kitti-Document_files/Kitti-Document_5_2.png}
    \par
    \end{center}
    
            \end{InvisibleVerbatim}
            
        
    
The accuracy of the model trained from scratch. It can be seen that is
dismal. Looking at rotation loss, especially for thetaY, it seems like
L1 penalty of currently used euclidean loss maybe a plausible way
forward.Initializing Training with AlexNet

One thing we notice is that

    % Make sure that atleast 4 lines are below the HR
    \needspace{4\baselineskip}

    
        \vspace{6pt}
        \makebox[0.1\linewidth]{\smaller\hfill\tt\color{nbframe-in-prompt}In\hspace{4pt}{[}32{]}:\hspace{4pt}}\\*
        \vspace{-2.65\baselineskip}
        \begin{ColorVerbatim}
            \vspace{-0.7\baselineskip}
            \begin{Verbatim}[commandchars=\\\{\}]
\PY{k+kn}{import} \PY{n+nn}{my\PYZus{}pycaffe} \PY{k+kn}{as} \PY{n+nn}{mp}
\PY{k+kn}{import} \PY{n+nn}{matplotlib.pyplot} \PY{k+kn}{as} \PY{n+nn}{plt}
\PY{n}{mp} \PY{o}{=} \PY{n+nb}{reload}\PY{p}{(}\PY{n}{mp}\PY{p}{)}
\PY{n}{ax1} \PY{o}{=} \PY{n}{plt}\PY{o}{.}\PY{n}{subplot}\PY{p}{(}\PY{l+m+mi}{2}\PY{p}{,}\PY{l+m+mi}{2}\PY{p}{,}\PY{l+m+mi}{1}\PY{p}{)}
\PY{n}{ax2} \PY{o}{=} \PY{n}{plt}\PY{o}{.}\PY{n}{subplot}\PY{p}{(}\PY{l+m+mi}{2}\PY{p}{,}\PY{l+m+mi}{2}\PY{p}{,}\PY{l+m+mi}{2}\PY{p}{)}
\PY{n}{ax3} \PY{o}{=} \PY{n}{plt}\PY{o}{.}\PY{n}{subplot}\PY{p}{(}\PY{l+m+mi}{2}\PY{p}{,}\PY{l+m+mi}{2}\PY{p}{,}\PY{l+m+mi}{3}\PY{p}{)}
\PY{n}{alexWeightFile} \PY{o}{=} \PY{l+s}{\PYZsq{}}\PY{l+s}{/data1/pulkitag/caffe\PYZus{}models/caffe\PYZus{}imagenet\PYZus{}train\PYZus{}iter\PYZus{}310000}\PY{l+s}{\PYZsq{}}
\PY{n}{weightFile10K}\PY{p}{,} \PY{n}{protoFile} \PY{o}{=} \PY{n}{ku}\PY{o}{.}\PY{n}{get\PYZus{}weight\PYZus{}proto\PYZus{}file}\PY{p}{(}\PY{n}{isScratch}\PY{o}{=}\PY{n+nb+bp}{False}\PY{p}{,} \PY{n}{numIter}\PY{o}{=}\PY{l+m+mi}{20000}\PY{p}{)}
\PY{n}{weightFile50K}\PY{p}{,} \PY{n}{protoFile} \PY{o}{=} \PY{n}{ku}\PY{o}{.}\PY{n}{get\PYZus{}weight\PYZus{}proto\PYZus{}file}\PY{p}{(}\PY{n}{isScratch}\PY{o}{=}\PY{n+nb+bp}{False}\PY{p}{,} \PY{n}{numIter}\PY{o}{=}\PY{l+m+mi}{50000}\PY{p}{)}
\PY{n}{netTune10K} \PY{o}{=} \PY{n}{mp}\PY{o}{.}\PY{n}{MyNet}\PY{p}{(}\PY{n}{protoFile}\PY{p}{,} \PY{n}{weightFile10K}\PY{p}{)}
\PY{n}{netTune10K}\PY{o}{.}\PY{n}{vis\PYZus{}weights}\PY{p}{(}\PY{l+s}{\PYZsq{}}\PY{l+s}{conv1}\PY{l+s}{\PYZsq{}}\PY{p}{,}\PY{n}{ax}\PY{o}{=}\PY{n}{ax1}\PY{p}{,} \PY{n}{titleName}\PY{o}{=}\PY{l+s}{\PYZsq{}}\PY{l+s}{Kitti\PYZhy{}10K Conv\PYZhy{}1 Weights}\PY{l+s}{\PYZsq{}}\PY{p}{)}
\PY{n}{netTune50K} \PY{o}{=} \PY{n}{mp}\PY{o}{.}\PY{n}{MyNet}\PY{p}{(}\PY{n}{protoFile}\PY{p}{,} \PY{n}{weightFile50K}\PY{p}{)}
\PY{n}{netTune50K}\PY{o}{.}\PY{n}{vis\PYZus{}weights}\PY{p}{(}\PY{l+s}{\PYZsq{}}\PY{l+s}{conv1}\PY{l+s}{\PYZsq{}}\PY{p}{,}\PY{n}{ax}\PY{o}{=}\PY{n}{ax2}\PY{p}{,} \PY{n}{titleName}\PY{o}{=}\PY{l+s}{\PYZsq{}}\PY{l+s}{Kitti\PYZhy{}50K Conv\PYZhy{}1 Weights}\PY{l+s}{\PYZsq{}}\PY{p}{)}
\PY{n}{netBase} \PY{o}{=} \PY{n}{mp}\PY{o}{.}\PY{n}{MyNet}\PY{p}{(}\PY{n}{protoFile}\PY{p}{,} \PY{n}{alexWeightFile}\PY{p}{)}
\PY{n}{netBase}\PY{o}{.}\PY{n}{vis\PYZus{}weights}\PY{p}{(}\PY{l+s}{\PYZsq{}}\PY{l+s}{conv1}\PY{l+s}{\PYZsq{}}\PY{p}{,}\PY{n}{ax}\PY{o}{=}\PY{n}{ax3}\PY{p}{,} \PY{n}{titleName}\PY{o}{=}\PY{l+s}{\PYZsq{}}\PY{l+s}{Base AlexNet Weights}\PY{l+s}{\PYZsq{}}\PY{p}{)}
\end{Verbatim}

            
                \vspace{-0.2\baselineskip}
            
        \end{ColorVerbatim}
    

    

        % If the first block is an image, minipage the image.  Else
        % request a certain amount of space for the input text.
        \needspace{4\baselineskip}
        
        

            % Add document contents.
            
                \begin{InvisibleVerbatim}
                \vspace{-0.5\baselineskip}
    \begin{center}
    \includegraphics[max size={\textwidth}{\textheight}]{Kitti-Document_files/Kitti-Document_8_0.png}
    \par
    \end{center}
    
            \end{InvisibleVerbatim}
            
        
    


    % Make sure that atleast 4 lines are below the HR
    \needspace{4\baselineskip}

    
        \vspace{6pt}
        \makebox[0.1\linewidth]{\smaller\hfill\tt\color{nbframe-in-prompt}In\hspace{4pt}{[}27{]}:\hspace{4pt}}\\*
        \vspace{-2.65\baselineskip}
        \begin{ColorVerbatim}
            \vspace{-0.7\baselineskip}
            \begin{Verbatim}[commandchars=\\\{\}]
\PY{n}{ku}\PY{o}{.}\PY{n}{get\PYZus{}accuracy}\PY{p}{(}\PY{n}{numIter}\PY{o}{=}\PY{l+m+mi}{20000}\PY{p}{,} \PY{n}{imSz}\PY{o}{=}\PY{l+m+mi}{256}\PY{p}{,} \PY{n}{poseType}\PY{o}{=}\PY{l+s}{\PYZsq{}}\PY{l+s}{euler}\PY{l+s}{\PYZsq{}}\PY{p}{,} \PY{n}{nrmlzType}\PY{o}{=}\PY{l+s}{\PYZsq{}}\PY{l+s}{zScoreScaleSeperate}\PY{l+s}{\PYZsq{}}\PY{p}{,}
                 \PY{n}{isScratch}\PY{o}{=}\PY{n+nb+bp}{False}\PY{p}{,} \PY{n}{concatLayer}\PY{o}{=}\PY{l+s}{\PYZsq{}}\PY{l+s}{pool5}\PY{l+s}{\PYZsq{}}\PY{p}{,} \PY{n}{numBatches}\PY{o}{=}\PY{l+m+mi}{3}\PY{p}{)}
\end{Verbatim}

            
                \vspace{-0.2\baselineskip}
            
        \end{ColorVerbatim}
    

    

        % If the first block is an image, minipage the image.  Else
        % request a certain amount of space for the input text.
        \needspace{4\baselineskip}
        
        

            % Add document contents.
            
                \begin{InvisibleVerbatim}
                \vspace{-0.5\baselineskip}
\begin{alltt}Intializing Network
Calculating Features
Plotting Results
\end{alltt}

            \end{InvisibleVerbatim}
            
                \begin{InvisibleVerbatim}
                \vspace{-0.5\baselineskip}
    \begin{center}
    \includegraphics[max size={\textwidth}{\textheight}]{Kitti-Document_files/Kitti-Document_9_1.png}
    \par
    \end{center}
    
            \end{InvisibleVerbatim}
            
                \begin{InvisibleVerbatim}
                \vspace{-0.5\baselineskip}
    \begin{center}
    \includegraphics[max size={\textwidth}{\textheight}]{Kitti-Document_files/Kitti-Document_9_2.png}
    \par
    \end{center}
    
            \end{InvisibleVerbatim}
            
        
    


    % Make sure that atleast 4 lines are below the HR
    \needspace{4\baselineskip}

    
        \vspace{6pt}
        \makebox[0.1\linewidth]{\smaller\hfill\tt\color{nbframe-in-prompt}In\hspace{4pt}{[}{]}:\hspace{4pt}}\\*
        \vspace{-2.65\baselineskip}
        \begin{ColorVerbatim}
            \vspace{-0.7\baselineskip}
            \begin{Verbatim}[commandchars=\\\{\}]

\end{Verbatim}

            
                \vspace{0.3\baselineskip}
            
        \end{ColorVerbatim}
    

        

        \renewcommand{\indexname}{Index}
        \printindex

    % End of document
    \end{document}


